\documentclass[fontsize=12pt, a4paper, headinclude, twoside=false, parskip=half+, pagesize=auto, numbers=noenddot, plainheadsepline, open=right, toc=listof, toc=bibliography, chapteratlists=0pt]{scrreprt}

% Allgemeines
\usepackage{scrlayer-scrpage} % Kopf- und Fußzeilen
\usepackage{amsmath,marvosym} % Mathesachen
\usepackage[T1]{fontenc} % Ligaturen, richtige Umlaute im PDF
\usepackage[utf8]{inputenc}% UTF8-Kodierung für Umlaute usw
\usepackage{caption}
\usepackage[Q=yes]{examplep}
\usepackage[colorinlistoftodos, textwidth=\marginparwidth]{todonotes}
\let\oldmissingfigure\missingfigure % save old command
\renewcommand{\missingfigure}[1]{\oldmissingfigure[figwidth=\textwidth-2pt]{#1}}% renew \missingfigure command
\usepackage{geometry}
\geometry{
	left = 30mm,
	right = 30mm,
	top = 28mm,
}
%\usepackage{showframe} %Zeigt Rahmen um alle Objekte an

% Schriften
\usepackage{setspace} % Zeilenabstand
\onehalfspacing % 1,5 Zeilen
\usepackage{lmodern}

% Schriften-Größen
\setkomafont{chapter}{\Huge\rmfamily} % Überschrift der Ebene
\setkomafont{section}{\Large\rmfamily}
\setkomafont{subsection}{\large\rmfamily}
\setkomafont{subsubsection}{\small\rmfamily}
\setkomafont{chapterentry}{\large\rmfamily} % Überschrift der Ebene in Inhaltsverzeichnis
\setkomafont{descriptionlabel}{\bfseries\rmfamily} % für description Umgebungen
\setkomafont{captionlabel}{\small\bfseries}
\setkomafont{caption}{\small}

% Sprache: Deutsch
\usepackage[ngerman]{babel} % Silbentrennung

% PDF
\usepackage[ngerman, breaklinks=true]{hyperref}
\usepackage[final]{microtype} % mikrotypographische Optimierungen
\usepackage{url}
\usepackage{pdflscape} % einzelne Seiten drehen können

% Tabellen
\usepackage{multirow} % Tabellen-Zellen über mehrere Zeilen
\usepackage{multicol} % mehre Spalten auf eine Seite
\usepackage{tabularx} % Für Tabellen mit vorgegeben Größen
\usepackage{longtable} % Tabellen über mehrere Seiten
\usepackage{array}
\usepackage{float}
\usepackage{booktabs}

% Bibliographie / Quellenverzeichnis
\usepackage{bibgerm} % Umlaute in BibTeX
\usepackage[style=alphabetic]{biblatex} % Quellenverzeichnis und Zitate
\addbibresource{bibliographie.bib}

% Bilder
\usepackage{graphicx} % Bilder
\graphicspath{{images/}}
\DeclareGraphicsExtensions{.pdf,.png,.jpg} % bevorzuge pdf-Dateien
\usepackage[all]{hypcap} % Beim Klicken auf Links zum Bild und nicht zu Caption gehen


% Bildunterschrift
\usepackage{caption}
\usepackage{chngcntr}
\counterwithout{figure}{chapter}
\setcapindent{0em} % kein Einrücken der Caption von Figures und Tabellen
%\setcapwidth[c]{0.9\textwidth}
\setlength{\abovecaptionskip}{0.2cm} % Abstand der zwischen Bild- und Bildunterschrift

% Custom colors
\definecolor{deepblue}{rgb}{0,0,0.5}
\definecolor{purple}{rgb}{0.96,0.15,0.44}
\definecolor{deepgreen}{rgb}{0,0.5,0}
\definecolor{codegreen}{rgb}{0,0.6,0}
\definecolor{codegray}{rgb}{0.4,0.4,0.4}
\definecolor{codeblue}{rgb}{0.16,0.32,0.75}
\definecolor{backcolour}{rgb}{0.95,0.95,0.92}
\definecolor{codeorange}{rgb}{1.0,0.49,0.0}

% Quellcode
\usepackage{listings} % für Formatierung in Quelltexten
\usepackage{color}
\lstdefinestyle{mystyle}{
	backgroundcolor=\color{backcolour},
	commentstyle=\color{codegray},
	keywordstyle=\color{purple},
	emph={instr,import},
	emphstyle={\color{codeorange}},
	numberstyle=\tiny\color{codegray},
	stringstyle=\color{codeblue},
	basicstyle=\footnotesize,
	breakatwhitespace=false,
	breaklines=true,
	captionpos=b,
	keepspaces=true,
	numbers=left,
	numbersep=10pt,
	showspaces=false,
	showstringspaces=false,
	showtabs=false,
	tabsize=2,
	otherkeywords={vec4},
	morekeywords={vec4},
	frame=single,
	framerule=2pt,
	rulecolor=\color{backcolour},
}
\lstset{style=mystyle,language=Python}

% Eigene Befehle %%%%%%%%%%%%%%%%%%%%%%%%%%%%%%%%%%%%%%%%%%%%%%%%%
\newcommand{\image}[4][!h]{
	\begin{figure}[#1]
		\centering
		\vspace{1ex}
		\includegraphics[#3]{images/#2}
		\caption[#4]{#4}\label{img.#2}      
		\vspace{1ex}
	\end{figure}
}
